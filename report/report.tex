\documentclass[a4paper,10pt]{article}

\usepackage[BoldFont,SlantFont,CJKsetspaces,CJKchecksingle]{xeCJK}
\usepackage{fancyhdr}
\usepackage{amsmath}
\usepackage{amssymb}
\usepackage{amsthm}
\usepackage{multirow}
\usepackage{longtable}
\usepackage{graphicx}
\usepackage{xcolor}
\usepackage{float}
\usepackage{tikz,pgfplots}
\usepackage{pifont}
\usetikzlibrary{arrows,scopes,svg.path,shapes,shadows,positioning,decorations.markings}
\usepackage{verbatim}
\usepackage[colorlinks,linkcolor=black,anchorcolor=blue,citecolor=green,filecolor=blue,urlcolor=blue]{hyperref}
\usepackage{listings}
\usepackage{setspace}
\usepackage{indentfirst}
\usepackage{cleveref}
\usepackage{mathrsfs}

\parindent 2em

\setCJKmainfont[BoldFont=SimHei,ItalicFont=KaiTi_GB2312]{SimSun}
\setCJKmonofont{SimSun}

\newtheorem{lemma}{\textbf{引理}}
\newtheorem{theorem}{\textbf{定理}}


\title{拼音输入法实验报告}

\author{计64~~陶东来~~~学号:2016011322}

\begin{document}
  \maketitle
  \section{算法基本思路}

  显然,在二元文法的前提下,当前这个拼音$p_i$所对应的汉字$x_i$只需考虑其前一个字$x_{i-1}$及其对应的拼音$p_{i-1}$即可。首先不考虑拼音,我们有:

  \[
    P(\Pi_{j=1}^{i} x_{j}) = P(\Pi_{j=1}^{i-1} x_{j}) P(x_{i}|x_{i-1})
  \]

  接下来考虑拼音的影响。在二元文法的假设下,有:

  \[
  P(\Pi_{j=1}^{i} x_{j} | \Pi_{j=1}^{i} w_j) = P(\Pi_{j=1}^{i-1} x_{j} | \Pi_{j-1}^{i-1} w_j) P(x_{i}|x_{i-1} w_{i} w_{i-1})
  \]

  因此现在的关键就变成了计算$P(x_{i}|x_{i-1} w_{i} w_{i-1})$。只要有了这个概率,我们就可以使用Viterbi algorithm求出最后的结果了。

  \subsection{Naive version}
  我们先“假装”$P(x_{i}|x_{i-1} w_{i} w_{i-1})=P(x_{i} | x_{i-1})$,由此实现程序的第一个版本,之后所有的修改都会基于这一版本。

  在这种情况下,我们只需简单地统计所有的字符二元组即可。大约数分钟即可完成数据的收集,并且可以应付很多情况。如"zhong hua ren min gong he guo",已经可以
  得出“中华人民共和国”这样我们想要的结果了。

  但是有一个非常严峻的问题,就是先前我们的假设是显然有问题的,最鲜明的例子就是"mo fa",这个版本的程序会给出“无法”。这是因为“无”是一个多音字,而实际上“无法”这个词
  真正应该归类在“wu fa”而不是“mo fa”。这就体现出了先前我们假想的问题,即“多音字分流”。

  \subsection{Improved version}

  我们考虑对多音字分流这个问题进行修正。事实上,只要我们将多音字的不同读音视作不同的字,那么先前的假设就是正确的了。因此,我们考虑对原文进行注音,那么之后的部分就
  和Naive version类似了。

  这是一个非常美好的设想。但是由于手头没有词语-拼音的词典,我无法自己进行注音,只能通过注音库来完成这一工作。在这里我使用的是pypinyin这个包。它的工作原理是先使用
  分词库对原文进行分词,然后再对照词典进行注音。这样带来的结果是我单单想要对那个最小的文件(20M)进行注音就要花费无法承受的时间。并且,我无法对这个库本身进行任何优化。

  \subsection{Final version}

  因此我选择退而求其次,对所有包含多音字的语段,取出1\%进行注音。这样,由于语料库本身足够大,这样抽取出的信息也足够获得$P(x_{i}|x_{i-1} w_{i} w_{i-1})$。
  尽管如此,在我使用了多线程并行的前提下,还是花费了1.5小时才处理完了所有的语料。

  在加上线性平滑和对同学们建立的测试样例集的支持,现在的版本已经可以达到很好地效果了。

  当然仍有不足。比如“qin dian ta dang xia yi ren”,我的输入法会给出“钦点他当下一人”,其实我们想要的是“钦点他当下一任”。只不过这个问题恐怕就算是基于字的三元模型也
  不见得能解决,要基于词的模型才能解决。但是对于所有文本的分词由之前看来,时间无法承受;并且我们也没有词典,想要获得词我们只能通过分词实现。而且考虑到注音的过程十分耗时,
  如果使用三元文法我会不得不提高注音的采样比例,这样不仅会导致模型膨胀,而且会导致训练时间进一步增长。这两点使得我暂时放弃了在模型层面的改进。

  而对于平滑取的系数来说,我第一次取就取到了不错的效果。之后并没有多做改动。

  \subsection{收获}

  本次实验让我熟悉了python和多线程通信编程技巧,了解并实现了Viterbi algorithm和其他一些概率模型。 这个实验可以说是人工智能中非常简单的一部分,
  以致现在普通人并不会将它算作人工智能。尽管如此,整个过程依然非常精彩有趣。

  希望之后会有其他有意思的实验:-)。
\end{document}
